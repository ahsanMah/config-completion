\section{Introduction}

A network’s backbone is its routing control plane; a set of rules and distributed routing protocols that describe how the network should operate. A control plane is thus defined through configuration files present on every individual routing device in the network. These configurations are written in vendor specific languages (e.g. Cisco and Juniper) and describe very low level behaviours of a particular router. Network operators tasked to configure control planes may also be required to satisfy various ‘policies’ that the owning organization wants to enforce: e.g certain devices should always be blocked from communicating with higher privileged devices.\\

Research has shown that configuring control planes can be extremely complex in modern networks~\cite{complexity}. Consequently, this causes configurations to be prone to errors, most of which are only uncovered during operation after a failure has already dealt significant damage~\cite{errors}. For example, in 2012, failure of a router in a Microsoft Azure data center triggered previously unknown configuration errors on other devices, degrading service in the West Europe region for over two hours~\cite{azure}. Similarly in 2017, Google made a small error in a protocol configuration which interrupted Japan's Internet for several hours~\cite{googlebgp}. These examples highlight a need to develop highly resilient configurations that perform reliably.\\  

Network operators thus try to minimize extraneous features by reusing existing configurations that have been known to work in the past. When creating a network, operators typically write templates containing specific configuration lines that define a base set of behaviours for different router roles~\cite{complexity}. These templates are then used to specialize individual routers to achieve objectives for their respective part of the network. Due to varying router specifications, the template systems used allow network operators to fill in parameters with appropriate information each time the template is used.\\ 


Writing templates, however, can be an inefficient solution when dealing with special cases that deviate greatly from the predefined archetypal configurations. We thus propose a different approach that can serve to complement existing techniques for writing routing configurations. We consider the problem of writing network configurations to be analogous to writing software code. Most configurations are written using vendor specific languages, that make use of rules and keywords similar to traditional programming languages. We envision an interactive system inspired by code completion engines that could be invoked by network operators as they are writing router configurations to offer them suggestions for what to put in next, or list the options available from the invocation point.\\ 


Recent research on software systems has shown that codebases tend to contain regularities, much like natural languages~\cite{naturalness}. This has motivated further research on using traditional Natural Language Processing techniques for code completion and token suggestion, resulting in fairly accurate models~\cite{naturalness, raychev}.We hypothesize a similar regularity for network configurations, especially since they tend to be homogeneous by design, reusing the same set of keywords/tokens. Some of our work over the summer tried to quantify this similarity between configurations. We analyzed router configurations from a large research university and calculated the average number of tokens shared by a particular router with the rest of the network. Our preliminary results showed that configurations shared between 85% and 99% of tokens across different routers. This prompted us to explore simple NLP techniques that could leverage these token similarities to produce useful suggestions or completions. We plan to train an n-gram model using existing configurations and evaluate the accuracy of the suggestions generated.