\section{Introduction}

Research has shown that configuring control planes can be extremely complex in modern networks~\cite{complexity}. Consequently, this causes configurations to be prone to errors, most of which are only uncovered during operation after a failure has already dealt significant damage~\cite{errors}. Network operators thus try to minimize errors by reusing existing configurations that have been known to work in the past. Most routers offer some form of simple built-in Command Line Interface (CLI), where operators can use vendor specific languages to update router configurations. Often, these CLIs will offer rudimentary tab completion, where they will alphabetically suggest all the options available for a token from the invocation point. These suggestions are often unhelpful as the user then has to search for the desired completion.\\ 

We propose a different approach that can serve to complement existing techniques for writing routing configurations by consider the problem of writing network configurations to be analogous to writing software code. Most configurations are written using vendor specific languages, that make use of rules and keywords similar to traditional programming languages. We envision an interactive system inspired by code completion engines that could be invoked by network operators as they are writing router configurations to offer them suggestions for what to put in next, or list the options available from the invocation point.\\ 


Recent research on software systems has shown that codebases tend to contain regularities, much like natural languages~\cite{naturalness}. This has motivated further research on using traditional Natural Language Processing techniques for code completion and token suggestion, resulting in fairly accurate models~\cite{naturalness, raychev}. We hypothesize a similar regularity for network configurations, especially since they tend to be homogeneous by design, reusing the same set of keywords/tokens. Our analysis of router configurations from a large research university showed that configurations shared between 85\% and 99\% of tokens across different routers. This prompted us to explore simple NLP techniques that could leverage these token similarities to produce useful suggestions or completions. Our preliminary results show that using an off-the-shelf NLP algorithm with minor modifications, can give us up to 93\% accuracy for some configurations.

